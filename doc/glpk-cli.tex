%* glpk-cli.tex *%

%***********************************************************************
%  This code is part of GLPK for C#/CLI.
%
%  Copyright (C) 2016 Heinrich Schuchardt,
%  <xypron.glpk@gmx.de>
%
%  GLPK for C#/CLI is free software: you can redistribute it and/or 
%  modify it under the terms of the GNU General Public License as
%  published by the Free Software Foundation, either version 3 of the
%  License, or (at your option) any later version.
%
%  GLPK for C#/CLI is distributed in the hope that it will be useful, but 
%  WITHOUT ANY WARRANTY; without even the implied warranty of 
%  MERCHANTABILITY or FITNESS FOR A PARTICULAR PURPOSE. See the GNU 
%  General Public License for more details.
%
%  You should have received a copy of the GNU General Public License
%  along with GLPK for C#/CLI. If not, see
%  <http://www.gnu.org/licenses/>.
%***********************************************************************

\documentclass[a4paper,11pt]{report}
\usepackage{hyperref}
\usepackage{parskip}
\usepackage{natbib}
\usepackage{url}
\usepackage{graphicx}
\usepackage{pdflscape}
\usepackage{xcolor}
\usepackage{listings}
\usepackage[top=2cm, bottom=2cm, left=2cm, right=2cm]{geometry}
\usepackage{makeidx}
%%generate index
\makeindex

\newcommand{\glpkCliVersion}{1.2.0}
\newcommand{\glpkVersionMajor}{4}
\newcommand{\glpkVersionMinor}{58}

\newcommand{\code}{\texttt}

\renewcommand\contentsname{\sf\bfseries Contents}
\renewcommand\chaptername{\sf\bfseries Chapter}
\renewcommand\appendixname{\sf\bfseries Appendix}

\setlength{\parindent}{0pt}
\setlength{\parskip}{10pt} 

\begin{document}
% Use C# style for listings.
% For escaping to latex inside listings use "~.".
\lstset{
  basicstyle=\ttfamily,
  showstringspaces=true,
  commentstyle=\color{blue},
  language=csh,
  escapeinside={~}{.}
}

\thispagestyle{empty}

\begin{center}

\vspace*{1in}

\begin{huge}
\sf\bfseries GNU Linear Programming Kit\linebreak
for C\# and the Common Language Interface
\end{huge}

\vspace{0.5in}

\begin{LARGE}
\sf Reference Manual
\end{LARGE}

\vspace{0.5in}

\begin{LARGE}
\sf Version \glpkCliVersion
\end{LARGE}

\vspace{0.5in}
\begin{Large}
\sf \today
\end{Large}
\end{center}

\newpage

\vspace*{1in}

\vfill

\medskip \noindent
Copyright \copyright{} 2015-{\the\year} Heinrich Schuchardt,
xypron.glpk@gmx.de

\medskip \noindent
Permission is granted to make and distribute verbatim copies of this
manual provided the copyright notice and this permission notice are
preserved on all copies.

\medskip \noindent
Permission is granted to copy and distribute modified versions of this
manual under the conditions for verbatim copying, provided also that the
entire resulting derived work is distributed under the terms of
a permission notice identical to this one.

\medskip \noindent
Permission is granted to copy and distribute translations of this manual
into another language, under the above conditions for modified versions.

\medskip \noindent
Windows is a registered trademark of Microsoft Corporation. 

\tableofcontents

\chapter{Introduction}
The GNU Linear Programming Kit (GLPK)\cite{GLPK} package supplies a solver for
large scale linear programming (LP) and mixed integer programming (MIP). The
GLPK project is hosted at
\linebreak\href{http://www.gnu.org/software/glpk}{http://www.gnu.org/software/glpk}.

It has two mailing lists:\index{support}
\begin{itemize}
\item\href{mailto:help-glpk@gnu.org}{help-glpk@gnu.org} and 
\item\href{mailto:bug-glpk@gnu.org}{bug-glpk@gnu.org}.
\end{itemize}
To subscribe to one of these lists, please, send an empty mail with a Subject:
header line of just "subscribe" to the list.

GLPK provides a library written in C and a standalone solver.

The source code provided at
\href{ftp://gnu.ftp.org/gnu/glpk/}{ftp://gnu.ftp.org/gnu/glpk/} contains the
documentation of the library in  file doc/glpk.pdf.

Project GLPK for C\#/CLI delivers a Common Language Interface binding for GLPK.
It is hosted at
\linebreak\href{http://glpk-cli.sourceforge.net/}{http://glpk-cli.sourceforge.net/}.

To report problems and suggestions concerning GLPK for C\#/CLI, please, send an
email to the author at
\href{mailto:xypron.glpk@gmx.de}{xypron.glpk@gmx.de}\index{support}.

\chapter{Getting started}

This chapter will run you through the installation of GLPK for C\#/CLI and the
execution of a trivial example.

\section{Installation}
\subsection{Windows}

The following description assumes:

\begin{itemize}
\item You are using a 64-bit version of Windows. Replace folder name w64 by w32
if you are using a 32-bit version.
\item The current version of GLPK is \glpkVersionMajor.\glpkVersionMinor.
Please, adjust paths if necessary.
\item Your path for program files is "C:\textbackslash Program Files". Please,
adjust paths if necessary.
\end{itemize}

Download the current version of GLPK for Windows from
\href{https://sourceforge.net/projects/winglpk/}{https://sourceforge.net/projects/winglpk/}.

The filename for version \glpkVersionMajor.\glpkVersionMinor\ is
winglpk-\glpkVersionMajor.\glpkVersionMinor.zip. Unzip the file. Copy folder
glpk-\glpkVersionMajor.\glpkVersionMinor\ to "C:\textbackslash Program
Files\textbackslash GLPK\textbackslash ".

To check the installation run the following command:

\lstset{language=bash,escapeinside={~}{.}}
\begin{lstlisting}
"C:\Program Files\GLPK\w64\glpsol.exe" --version
\end{lstlisting}

To use GLPK for C\#/CLI you need a .NET environment to be installed.

\subsection{Linux}

Download the current version of GLPK source with

\begin{lstlisting}
wget ftp://ftp.gnu.org/gnu/glpk/glpk-~\glpkVersionMajor..~\glpkVersionMinor..tar.gz
\end{lstlisting}

Unzip the archive with:

\begin{lstlisting}
tar -xzf glpk-~\glpkVersionMajor..~\glpkVersionMinor..tar.gz
cd glpk-~\glpkVersionMajor..~\glpkVersionMinor.
\end{lstlisting}

Configure with

\begin{lstlisting}
./configure
\end{lstlisting}

Make and install with:

\begin{lstlisting}
make
make check
sudo make install
sudo ldconfig
\end{lstlisting}

Check the installation with

\begin{lstlisting}
glpsol --version
\end{lstlisting}

For the next steps you will need the Mono C\# compiler to be installed.

You can check the correct installation with the following commands:

\begin{lstlisting}
mcs --version
\end{lstlisting}

To build GLPK for C\#/CLI you will need package SWIG (Simplified Wrapper and
Interface Generator, \href{http://www.swig.org/}{http://www.swig.org/}). You
can check the installation with the following command:

\begin{lstlisting}
swig -version
\end{lstlisting}

Most Linux distribution contain a SWIG package. The installation command will
depend on the distribution, e.g.

\begin{itemize}
\item Debian: sudo apt-get install swig
\item Fedora: sudo yum install swig
\item Gentoo: sudo emerge swig
\end{itemize}

Download GLPK for C\#/CLI from
\href{https://sourceforge.net/projects/glpk-cli/files/}{https://sourceforge.net/projects/glpk-cli/}.

Unzip the archive with:

\begin{lstlisting}
tar -xzf glpk-cli-~\glpkCliVersion..tar.gz
cd glpk-cli-~\glpkCliVersion.
\end{lstlisting}

Configure with:
\begin{lstlisting}
./configure
\end{lstlisting}

If libglpk.so is in a special path you may specify this path using parameter
LDFLAGS, e.g.

\begin{lstlisting}
./configure LDFLAGS=-L/opt/lib
\end{lstlisting}

Make and install with:

\begin{lstlisting}
make
make check
sudo make install
sudo ldconfig
\end{lstlisting}

\section{Trivial example}

In the example we will create a C\# class which will write the GLPK version to
the console.

With a text editor create a text file test.cs with the following content:

\lstset{language=csh,escapeinside={~}{.}}
\begin{lstlisting}
using System;
using org.gnu.glpk;

class Program
{
  static void Main (string[] args)
  {
    Console.WriteLine ("GLPK " + GLPK.glp_version ());
  }
}
\end{lstlisting}

\subsection{Windows}

Copy the GLPK for C\#/CLI assembly (libglpk-cli.dll) to the directory with file
test.cs.

For compiling you need the C\# compiler.
The location of csc.exe depends on the version of the .NET framework used.
Possible paths include:

\begin{itemize}
\item C:\textbackslash Windows\textbackslash Microsoft.NET\textbackslash Framework64\textbackslash v4.0.30319\textbackslash 
\item C:\textbackslash Windows\textbackslash Microsoft.NET\textbackslash Framework64\textbackslash v3.5\textbackslash 
\item C:\textbackslash Windows\textbackslash Microsoft.NET\textbackslash Framework64\textbackslash v2.0.50727\textbackslash 
\end{itemize}

Compile the application:

\lstset{language=bash,escapeinside={~}{.}}
\begin{lstlisting}
csc.exe /r:libglpk-cli.dll test.cs
\end{lstlisting}

To compile a 32 bit application on 64 bit Windows use:

\lstset{language=bash,escapeinside={~}{.}}
\begin{lstlisting}
csc.exe /platform:x86 /r:libglpk-cli.dll test.cs
\end{lstlisting}

For running the application both the GLPK library
(glpk\_\glpkVersionMajor\_\glpkVersionMinor.dll) and the GLPK for C\#/CLI
native library (libglpk\_cli\_native.dll) have to be in the search path.
Set the environment variable PATH accordingly.

\lstset{language=bash,escapeinside={~}{.}}
\begin{lstlisting}
SET PATH=%PATH%;C:\Program Files\glpk\glpk-~\glpkVersionMajor..~\glpkVersionMinor.\w64
\end{lstlisting}

Run the application

\begin{lstlisting}
test.exe
\end{lstlisting}

The output will be the GLPK version number, for example:
GLPK \glpkVersionMajor.\glpkVersionMinor.

\subsection{Linux}
Compile the class

\begin{lstlisting}
mcs -r:libglpk-cli -lib:/usr/local/lib/glpk-cli/ test.cs
\end{lstlisting}

Run the application:

\begin{lstlisting}
export MONO_PATH=/usr/local/lib/glpk-cli
export LD_LIBRARY_PATH=/usr/local/lib/glpk-cli
./test.exe
\end{lstlisting}

The output will be the GLPK version number, for example:
GLPK \glpkVersionMajor.\glpkVersionMinor.

\chapter{Architecture}
A GLPK for C\#/CLI application will consist of
\begin{itemize}
\item the GLPK library
\item the GLPK for C\#/CLI library
\item the GLPK for C\#/CLI assembly
\item the application code.
\end{itemize}

\chapter{Classes}
\index{classes}
GLPK for C\#/CLI uses the Simplified Wrapper and Interface Generator
(SWIG)\index{SWIG}\cite{SWIG} to create
the Common Language Inteface binding for GLPK.
\index{class path}
Classes are created in namespace org.gnu.glpk.

Class GlpkCallback is called by the MIP solver callback routine.

Interface IGlpkCallbackListener can be implemented to register a listener for
class GlpkCallback.

Class GlpkTerminal is called by the MIP solver terminal output routine.

Interface IGlpkTerminalListener can be implemented to register a listener for
class GlpkTerminal.

Class GlpkException is thrown if an error occurs.

Class GLPK maps the functions and constants from glpk.h.

The following classes map structures from glpk.h:
\begin{itemize}
\item glp\_arc
\item glp\_attr
\item glp\_bfcp
\item glp\_cpxcp
\item glp\_graph
\item glp\_iocp
\item glp\_iptcp
\item glp\_long
\item glp\_mpscp
\item glp\_prob
\item glp\_smcp
\item glp\_tran
\item glp\_tree
\item glp\_vertex
\end{itemize}

The following classes are used to map pointers:
\begin{itemize}
\item SWIGTYPE\_p\_double
\item SWIGTYPE\_p\_f\_p\_q\_const\_\_char\_v\_\_\_\_\_\_\_void
\item SWIGTYPE\_p\_f\_p\_struct\_glp\_tree\_p\_void\_\_void
\item SWIGTYPE\_p\_f\_p\_void\_\_void
\item SWIGTYPE\_p\_f\_p\_void\_p\_q\_const\_\_char\_\_int
\item SWIGTYPE\_p\_glp\_arc
\item SWIGTYPE\_p\_glp\_graph
\item SWIGTYPE\_p\_glp\_vertex
\item SWIGTYPE\_p\_int
\item SWIGTYPE\_p\_p\_char
\item SWIGTYPE\_p\_p\_glp\_vertex
\item SWIGTYPE\_p\_size\_t
\item SWIGTYPE\_p\_va\_list
\item SWIGTYPE\_p\_void
\end{itemize}

The following clases are used for network problems:
\begin{itemize}
\item glp\_cli\_arc\_data
\item glp\_cli\_vertex\_data
\end{itemize}

\section{Exceptions}
\index{exceptions}
\index{GlpkException}
When illegal parameters are passed to a function of the GLPK native library
an exception GlpkException is thrown. Due to the architecture of GLPK all
GLPK objects are invalid when such an exception has occured.

\subsection{Implementation details}
GLPK for C\#/CLI registers a function glp\_cli\_error\_hook() to glp\_error\_hook()
before calling an GLPK API function. If an error occurs function glp\_free\_env
is called and a long jump is used to return to the calling environment. Then
function glp\_cli\_throw() is called which throws GlpkException.

\section{Network problems}
For network problems additional data like capacity and cost of arcs or the
inflow of vertics has to be specified. The GLPK library does not provide
data structures. In GLPK for C\#/CLI classes \_glp\_cli\_arc\_data and
\_glp\_cli\_vertex\_data are provided.

When creating a graph the size of the structures for these classes has to be
specified. In some routines the offsets to individual fields in the structures are
needed. The following constants have been defined:

\begin{itemize}
\item GLP\_CLI\_A\_CAP  - offset of field cap in arc data
\item GLP\_CLI\_A\_COST - offset of field cost in arc data
\item GLP\_CLI\_A\_LOW  - offset of field low in arc data
\item GLP\_CLI\_A\_RC   - offset of field rc in arc data
\item GLP\_CLI\_A\_X    - offset of field x in arc data
\item GLP\_CLI\_A\_SIZE - size of arc data
\item GLP\_CLI\_V\_CUT  - offset of field cut in vertex data
\item GLP\_CLI\_V\_PI   - offset of field pi in vertex data
\item GLP\_CLI\_V\_RHS  - offset of field rhs in vertex data
\item GLP\_CLI\_V\_SET  - offset of field set in vertex data
\item GLP\_CLI\_V\_SIZE - size of vertex data
\end{itemize}

For accessing vertices method GLPK.glp\_cli\_vertex\_get can be used.

For accessing the data areas of arcs and vertices methods
\begin{itemize}
\item GLPK.glp\_cli\_arc\_get\_data,
\item GLPK.glp\_cli\_vertex\_data\_get, and
\item GLPK.glp\_cli\_vertex\_get\_data
\end{itemize}
can be used.

\lstset{language=csh,escapeinside={~}{'}}
\begin{lstlisting}
glp_arc arc;
glp_cli_arc_data adata;
glp_cli_vertex_data vdata;

glp_graph graph =
        GLPK.glp_create_graph(
        GLPKConstants.GLP_CLI_V_SIZE,
        GLPK.GLP_CLI_A_SIZE);
GLPK.glp_set_graph_name(graph,
        MinimumCostFlow.class.getName());

int ret = GLPK.glp_add_vertices(graph, 9);

GLPK.glp_set_vertex_name(graph, 1, "v1");
GLPK.glp_set_vertex_name(graph, 2, "v2");
GLPK.glp_set_vertex_name(graph, 3, "v3");
GLPK.glp_set_vertex_name(graph, 4, "v4");
GLPK.glp_set_vertex_name(graph, 5, "v5");
GLPK.glp_set_vertex_name(graph, 6, "v6");
GLPK.glp_set_vertex_name(graph, 7, "v7");
GLPK.glp_set_vertex_name(graph, 8, "v8");
GLPK.glp_set_vertex_name(graph, 9, "v9");

vdata = GLPK.glp_cli_vertex_data_get(graph, 1);
vdata.setRhs(20);
vdata = GLPK.glp_cli_vertex_data_get(graph, 9);
vdata.setRhs(-20);

arc = GLPK.glp_add_arc(graph, 1, 2);
adata = GLPK.glp_cli_arc_get_data(arc);
adata.setLow(0); adata.setCap(14); adata.setCost(0);

...

GLPK.glp_write_mincost(graph,
        GLPKConstants.GLP_CLI_V_RHS,
        GLPKConstants.GLP_CLI_A_LOW,
        GLPKConstants.GLP_CLI_A_CAP,
        GLPKConstants.GLP_CLI_A_COST,
        "mincost.dimacs");
GLPK.glp_delete_graph(graph);

\end{lstlisting}

\section{Callbacks}
\index{callbacks}
\index{GlpkCallback}
\index{IGlpkCallbackListener}
The MIP solver provides a callback functionality. This is used to call
method callback of class GlpkCallback. A CLI program can listen to the
callbacks by instantiating a class implementing interface
IGlpkCallbackListener and registering the object with method addListener()
of class GlpkCallback. The listener can be deregistered with method
removeListener(). The listener can use method GLPK.glp\_ios\_reason() to find
out why it is called. For details see the GLPK library documentation.

\section{Output listener}
\index{output listener}
\index{GlpkTerminal}
\index{IGlpkTerminalListener}
GLPK provides a hook for terminal output. A CLI program can listen to the
callbacks by instantiating a class implementing interface IGlpkTerminalListener
and registering the object with method addListener of class GlpkTerminal.
The listener can be dregistered with method removeListener().
After a call to glp\_free\_env() the GlpkTerminal has to registered again
by calling GLPK.glp\_term\_hook(null, null). glp\_free\_env() is called if
an exception GlpkException occurs.

\section{Aborting a GLPK library call}
\index{abort}
\index{GlpkException}
\index{glp\_cli\_error}
Method void GLPK.glp\_cli\_error(String message) can be used to abort any call
to the GLPK library. An exception GlpkException will occur. As GLPK is not
threadsafe the call must be placed in the same thread as the initial call that
is to be aborted. The output method of a GlpkTerminalListener can be used
for this purpose.

\section{Debugging support}
\index{message level}
\index{debug}
\index{glp\_cli\_set\_msg\_lvl}
Method void GLPK.glp\_cli\_set\_msg\_lvl(int msg\_lvl) can be used to enable
extra output signaling when a GLPK library function is entered or left using 
value with GLPKConstants.GLP\_CLI\_MSG\_LVL\_ALL. The output is disabled by a
call with value GLPKConstants.GLP\_CLI\_MSG\_LVL\_OFF.

\section{Locales}
\index{locales}
\index{glp\_cli\_set\_numeric\_locale}
Method void GLPK.glp\_cli\_set\_numeric\_locale(String locale) can be used to
set the locale for numeric formatting. When importing model files the GLPK
library expects to be using locale "C".

\section{Threads}
\index{threads}
The GLPK library is not thread safe. Never two threads should be running that
access the GLPK library at the same time. When a new thread accesses the
library it should call GLPK.glp\_free\_env(). When using an GlpkTerminalListener
it is necessary to register GlpkTerminal again by calling
\linebreak GLPK.glp\_term\_hook(null, null).

When writing a GUI application it is advisable to use a separate thread for
the calls to GLPK. Otherwise the GUI cannot react to events during the call
to the GLPK libary.

\chapter{License}
\index{license}
GLPK for C\#/CLI is free software: you can redistribute it and/or
modify it under the terms of the GNU General Public License\cite{GPL} as
published by the Free Software Foundation, either version 3 of the
License, or (at your option) any later version.

GLPK for C\#/CLI is distributed in the hope that it will be useful, but
WITHOUT ANY WARRANTY; without even the implied warranty of
MERCHANTABILITY or FITNESS FOR A PARTICULAR PURPOSE. See the GNU
General Public License for more details.

You should have received a copy of the GNU General Public License
along with GLPK for C\#/CLI. If not, see
\href{http://www.gnu.org/licenses/}{http://www.gnu.org/licenses/}.

\bibliographystyle{plain}
\bibliography{mybib}
\newpage
\printindex
\end{document}
